\documentclass[a4paper,11pt,oneside]{article}

\usepackage{amsmath,amssymb,epsfig}
\usepackage[T1]{fontenc}
\usepackage{ae,aecompl}
\usepackage{url}
\usepackage{subfigure}
\addtolength{\voffset}{-1cm}
\addtolength{\hoffset}{-1cm}
\setlength{\parindent}{0in}
\addtolength{\textwidth}{1.8cm}
\addtolength{\textheight}{1cm}
\addtolength{\parskip}{.5cm}

% Example definitions.
% --------------------
\def\x{{\mathbf x}}
\def\X{{\mathbf X}}
\def\u{{\mathbf u}}
\def\U{{\mathbf U}}
\def\x{{\mathbf x}}
\def\s{{\mathbf s}}
\def\A{{\mathbf A}}
\def\y{{\mathbf y}}
\def\W{{\mathbf W}}
\def\w{{\mathbf w}}
\def\B{{\mathbf B}}
\def\a{{\mathbf a}}
\def\D{{\mathbf D}}
\def\P{{\mathbf P}}
\def\n{{\mathbf n}}
\def\V{{\mathbf V}}
\def\R{{\mathbf R}}
\def\I{{\mathbf I}}
\def\M{{\mathbf M}}
\def\sech{{\mathrm{sech}}}
\def\L{{\cal L}}
\def\Cum{{\rm{Cum}}}
\def\var{{\rm{var}}}
\def\T{{\mathbf T}}
\def\C{{\mathbf C}}
\def\tf{{\emph{t-f}}}


% Title.
% ------
\title{\large{\textbf{EXERCISE 4}}}
\author{SGN-1156 Signal Processing Techniques\\
\url{http://www.cs.tut.fi/courses/SGN-1156}\\
Tampere University of Technology\\
Germ\'an G\'omez-Herrero, \url{http://germangh.com}}
\date{November 23, 2009}


\begin{document}
\maketitle

\noindent \textbf{PROBLEM 1 (problem 5.21 from the book):} Let $x[n]$, $0\leq n \leq N-1$ be a length-$N$ sequence with an $N$-point DFT $X[k]$, $0 \leq k \leq N-1$. Determine the $N$-point DFTs of the following length-$N$ sequences in terms of $X[k]$:

\begin{itemize}
\item[(a)] $w[n]=\alpha x[\left\langle n-m_1\right\rangle_{N}]+\beta x[\left\langle n-m_2\right\rangle_{N}]$, where $m_1$ and $m_2$ are positive integers less than $N$.
\item[(b)] $g[n]=\left\{\begin{array}{lll}x[n]&\quad&\textrm{for } n \textrm{ even}\\0&\quad&\textrm{for } n \textrm{ odd}\end{array}\right.$
\item[(c)] $y[n]=x[n]\stackrel{N}{\otimes} x[n]$. 
\end{itemize}


\vspace{1cm}

\noindent \textbf{PROBLEM 2 (problem 5.23 from the book):} Let $x[n]$, $0\leq n \leq N-1$ be a length-$N$ sequence with an $N$-point DFT $X[k]$, $0\leq k \leq N-1$. Determine the $N$-point inverse DFTs of the following length-$N$ DFTs in terms of $x[n]$:

\begin{itemize}
\item[(a)] $W[k]=\alpha X[\left\langle k-m_1\right\rangle_{N}]+\beta x[\left\langle k-m_2\right\rangle_{N}]$, where $m_1$ and $m_2$ are positive integers less than $N$.
\item[(b)] $G[k]=\left\{\begin{array}{lll}X[k]&\quad&\textrm{for } k \textrm{ even}\\0&\quad&\textrm{for } k \textrm{ odd}\end{array}\right.$
\item[(c)] $Y[k]=X[k]\stackrel{N}{\otimes} X[k]$. 
\end{itemize}


\vspace{1cm}

\noindent \textbf{PROBLEM 3 (problem 5.42 from the book):} A 126-point DFT $X[k]$ of a real-valued sequence $x[n]$ has the following DFT samples: $X[0]=12.8+j\alpha$, $X[13]=-3.7+j2.2$, $X[k_1]=9.1-j5.4$, $X[k_2]=6.3+j2.3$, $X[51]=-j1.7$, $X[63]=13+j\beta$, $X[k_3]=\gamma+j1.7$, $X[79]=6.3+j\delta$, $X[108]=\epsilon+j5.4$, $X[k_4]=-3.7-j2.2$. The remaining DFT samples are assumed to be equal to zero.

\begin{itemize}
\item[(a)] Determine the values of the indices $k_1$, $k_2$, $k_3$ and $k_4$.
\item[(b)] Determine the values of $\alpha$, $\beta$, $\delta$, and $\epsilon$.
\item[(c)] What is the DC value of $\left\{x[n]\right\}$?
\item[(d)] Determine the expression for $\left\{x[n]\right\}$ without computing the IDFT.
\item[(e)] What is the energy of $\left\{x[n]\right\}$?
\end{itemize}


\end{document}