\documentclass[a4paper,11pt,oneside]{article}

\usepackage{amsmath,amssymb,epsfig}
\usepackage[T1]{fontenc}
\usepackage{ae,aecompl}
\usepackage{url}
\usepackage{subfigure}
\addtolength{\voffset}{-1cm}
\addtolength{\hoffset}{-1cm}
\setlength{\parindent}{0in}
\addtolength{\textwidth}{1.8cm}
\addtolength{\textheight}{1cm}
\addtolength{\parskip}{.5cm}

% Example definitions.
% --------------------
\def\e{{e^{j\omega}}}
\def\W{{W_M}}
\def\sumk{{\sum_{k=-\infty}^{\infty}}}
\def\x{{\mathbf x}}
\def\X{{\mathbf X}}
\def\u{{\mathbf u}}
\def\U{{\mathbf U}}
\def\x{{\mathbf x}}
\def\s{{\mathbf s}}
\def\A{{\mathbf A}}
\def\y{{\mathbf y}}
\def\W{{\mathbf W}}
\def\w{{\mathbf w}}
\def\B{{\mathbf B}}
\def\a{{\mathbf a}}
\def\D{{\mathbf D}}
\def\P{{\mathbf P}}
\def\n{{\mathbf n}}
\def\V{{\mathbf V}}
\def\R{{\mathbf R}}
\def\I{{\mathbf I}}
\def\M{{\mathbf M}}
\def\sech{{\mathrm{sech}}}
\def\L{{\cal L}}
\def\Cum{{\rm{Cum}}}
\def\var{{\rm{var}}}
\def\T{{\mathbf T}}
\def\C{{\mathbf C}}
\def\tf{{\emph{t-f}}}


% Title.
% ------
\title{\large{\textbf{EXERCISE 4 - SOLUTIONS}}}
\author{SGN-1156 Signal Processing Techniques\\
\texttt{http://www.cs.tut.fi/courses/SGN-1156}\\
Tampere University of Technology\\
Germ\'an G\'omez-Herrero, \url{http://germangh.com}}

\date{November 25, 2009}

\begin{document}
\maketitle

\noindent \textbf{PROBLEM 1 (problem 5.21 from the book):} Let $x[n]$, $0\leq n \leq N-1$ be a length-$N$ sequence with an $N$-point DFT $X[k]$, $0 \leq k \leq N-1$. Determine the $N$-point DFTs of the following length-$N$ sequences in terms of $X[k]$:

\begin{itemize}
\item[(a)] $w[n]=\alpha x[\left\langle n-m_1\right\rangle_{N}]+\beta x[\left\langle n-m_2\right\rangle_{N}]$, where $m_1$ and $m_2$ are positive integers less than $N$.
\item[(b)] $g[n]=\left\{\begin{array}{lll}x[n]&\quad&\textrm{for } n \textrm{ even}\\0&\quad&\textrm{for } n \textrm{ odd}\end{array}\right.$
\item[(c)] $y[n]=x[n]\stackrel{N}{\otimes} x[n]$. 
\end{itemize}

\vspace{1cm}
%%%%%%%%%%%%%%%%%%%%%%%%%%%%%%%%%%%%%%%%%%%%%%%%%%%%%%%%%%%%%%%
\textbf{SOLUTION:}

\textbf{(a)}

Using the circular time-shifting property from Table 5.3 of the book:

\[
W[k] = \left(\alpha W_N^{m_1k}+\beta W_N^{m_2k}\right)X[k]
\]

\textbf{(b)}


\[
g[n] = \left(\frac{1}{2}+(-1)^n\right)x[n]=\left(\frac{1}{2}+W_N^{\frac{Nn}{2}}\right)x[n]
\]

\noindent So using the circular frequency-shifting property from Table 5.3 of the book:

\[
G[k]=\frac{1}{2}X[k]+\frac{1}{2}X\left[\langle k-\frac{N}{2}\rangle_N\right]
\]


\textbf{(c)}

Using the N-point circular convolution property from Table 5.3 of the book:

\[
Y[k] = X[k]X[k]=X^2[k]
\]



%%%%%%%%%%%%%%%%%%%%%%%%%%%%%%%%%%%%%%%%%%%%%%%%%%%%%%%%%%%%%%%
\vspace{1cm}

\noindent \textbf{PROBLEM 2 (problem 5.23 from the book):} Let $x[n]$, $0\leq n \leq N-1$ be a length-$N$ sequence with an $N$-point DFT $X[k]$, $0\leq k \leq N-1$. Determine the $N$-point inverse DFTs of the following length-$N$ DFTs in terms of $x[n]$:

\begin{itemize}
\item[(a)] $W[k]=\alpha X[\left\langle k-m_1\right\rangle_{N}]+\beta x[\left\langle k-m_2\right\rangle_{N}]$, where $m_1$ and $m_2$ are positive integers less than $N$.
\item[(b)] $G[k]=\left\{\begin{array}{lll}X[k]&\quad&\textrm{for } k \textrm{ even}\\0&\quad&\textrm{for } k \textrm{ odd}\end{array}\right.$
\item[(c)] $Y[k]=X[k]\stackrel{N}{\otimes} X[k]$. 
\end{itemize}




\vspace{1cm}
%%%%%%%%%%%%%%%%%%%%%%%%%%%%%%%%%%%%%%%%%%%%%%%%%%%%%%%%%%%%%%%
\textbf{SOLUTION:}

\textbf{(a)}

Using the circular frequency-shifting property from Table 5.3 of the book:

\[
w[n]=\left(\alpha W_{N}^{-m_1n}+\beta W_{N}^{-m_2n} \right)x[n]
\]


\textbf{(b)}

\[
G[k] = \left(\frac{1}{2}+(-1)^n\right)X[k]=\left(\frac{1}{2}+W_N^{\frac{Nk}{2}}\right)X[k]
\]

\noindent So using the circular time-shifting property from Table 5.3 of the book:

\[
g[n] = \frac{1}{2}x[n]+\frac{1}{2}x[\langle n-\frac{N}{2}\rangle_N]
\]


\textbf{(c)}

Using the modulation property from Table 5.3 of the book:

\[
y[n] = Nx[n]x[n]=Nx^2[n]
\]

%%%%%%%%%%%%%%%%%%%%%%%%%%%%%%%%%%%%%%%%%%%%%%%%%%%%%%%%%%%%%%%
\vspace{1cm}

\noindent \textbf{PROBLEM 3 (problem 5.42 from the book):} A 126-point DFT $X[k]$ of a real-valued sequence $x[n]$ has the following DFT samples: $X[0]=12.8+j\alpha$, $X[13]=-3.7+j2.2$, $X[k_1]=9.1-j5.4$, $X[k_2]=6.3+j2.3$, $X[51]=-j1.7$, $X[63]=13+j\beta$, $X[k_3]=\gamma+j1.7$, $X[79]=6.3+j\delta$, $X[108]=\epsilon+j5.4$, $X[k_4]=-3.7-j2.2$. The remaining DFT samples are assumed to be equal to zero.

\begin{itemize}
\item[(a)] Determine the values of the indices $k_1$, $k_2$, $k_3$ and $k_4$.
\item[(b)] Determine the values of $\alpha$, $\beta$, $\gamma$, $\delta$, and $\epsilon$.
\item[(c)] What is the DC value of $\left\{x[n]\right\}$?
\item[(d)] Determine the expression for $\left\{x[n]\right\}$ without computing the IDFT.
\item[(e)] What is the energy of $\left\{x[n]\right\}$?
\end{itemize}


\vspace{1cm}
%%%%%%%%%%%%%%%%%%%%%%%%%%%%%%%%%%%%%%%%%%%%%%%%%%%%%%%%%%%%%%%
\textbf{SOLUTION:}

\textbf{(a)}

Since $x[n]$ is real-valued we have the symmetry property $X[k]=X^*[\langle-k\rangle_N]$. This means that if $X[k_1]=9.1-j5.4$ then $X[\langle-k_1\rangle_N]=9.1+j5.4$. From the given values of $X[k]$ it is clear that $\langle-k_1\rangle_N=N-k_1=108\Rightarrow k_1=126-108=18$. Similarly we get:

\[
\langle-k_2\rangle_N=N-k_2=79\Rightarrow k_2=126-79=47
\] 

\[
\langle-k_3\rangle_N=N-k_3=51\Rightarrow k_3=126-51=75
\] 

\[
\langle-k_4\rangle_N=N-k_4=13\Rightarrow k_4=126-13=113
\] 



\textbf{(b)}

Again, using the symmetry $X[k]=X^*[\langle-k\rangle_N]$ we have that $X[0]=X^*[0]\Rightarrow \textrm{Im}\left\{X[0]\right\}=0\Leftrightarrow \alpha=0$. Similarly:

\[
\begin{array}{lll}
X[63]&=&13+j\beta = X^*[\langle -63\rangle_N]=X^*[N-63]=X^*[63] \Leftrightarrow \textrm{Im}\left\{X[63]\right\}=0 \Leftrightarrow \beta=0\\
X[75]&=&\gamma+j1.7=X^*[\langle-75\rangle_N]=X^*[N-75]=X^*[51]=j1.7\Leftrightarrow \gamma=0\\
X[79]&=&6.3+j\delta=X^*[\langle-79\rangle_N]=X^*[N-79]=X^*[47]=6.3-j2.3\Leftrightarrow \delta=-2.3\\
X[108]&=&\epsilon+j5.4=X^*[\langle-108\rangle_N]=X^*[N-108]=X^*[18]=9.1+j5.4\Leftrightarrow \epsilon=9.1\\
\end{array}
\]


\textbf{(c)}

The DC value is $X[0]=12.8$

\textbf{(d)}

\[
\begin{array}{lll}
x[n] &=& \frac{1}{126}\sum_{k=0}^{125}X[k]W_{126}^{-kn}=\frac{1}{126}\left(X[0]+X[63]W_{126}^{-63n}+2\textrm{Re}\left\{X[13]\right\}W_{N}^{-13n}+\right.\\
&&\left. 2\textrm{Re}\left\{X[18]\right\}W_{N}^{-18n}+2\textrm{Re}\left\{X[47]\right\}W_{N}^{-47n}+2\textrm{Re}\left\{X[75]\right\}W_{N}^{-75n} +\right.\\
&&\left.+2\textrm{Re}\left\{X[113]\right\}W_{N}^{-113n}\right)
\end{array}
\]

\textbf{(e)}


\[
E_x=\sum_{n=1}^{125}|x[n]|^2=\frac{1}{126}\sum_{k=0}^{125} |X[k]|^2=5.767
\]

%%%%%%%%%%%%%%%%%%%%%%%%%%%%%%%%%%%%%%%%%%%%%%%%%%%%%%%%%%%%%%%

\end{document}